\chapter{Creación del servidor local}

\section{Registro en ngrok}

Pasos para tener ngrok:

\begin{itemize}
    \item Instalar ngrok mediante snap
    
\begin{lstlisting}
snap install ngrok
\end{lstlisting} 

    \item Conectar ngrok a la cuenta
  
\begin{lstlisting}
ngrok authtoken 25Pt1y3GT9JEnLudmofnYx6xvrS_2D2gZBpDpQGLhipQSfiHN
\end{lstlisting}  


Este comando generará un archivo de configuración llamado ngrok.yml

    \item Reiniciar ngrok
    
Con el reinicio se harán efectivos los cambios.
    
\end{itemize}

\section{Instalación requisitos}

La instalación está detallada en el Readme

\section{Creación del servidor}

Pasos:

\begin{itemize}
    \item Generación del archivo package
    
\begin{lstlisting}
$ npm init 
\end{lstlisting}

Este archivo package contendrá toda la información sobre la configuración del servidor. Se ejecuta el comando y se acepta toda la configuración por defecto que se va mostrando por pantalla.
    
    \item (Opcional) Modificar la configuración
    
Se modifica el archivo package, generado en el anterior paso,rellenando los campos dejados por defecto.

    \item Creación del archivo index.js
    
En este primer archivo index.js he copiado las líneas de código que aparecen en la página de descarga de express (referencia a https://www.npmjs.com/package/express).

\begin{lstlisting}
const express = require('express')
const app = express()

app.get('/', function (req, res) {
  res.send('Hello World')
})

app.listen(puerto, () => {
    console.log("Estamos ejecutando el servidor en el puerto " + puerto)
});
\end{lstlisting}

Este será el código base sobre el que empezaré a crear mi servidor local.

    \item Comprobación de conexión con el servidor
    
Para comprobar si el puerto no está siendo usado por otro servicio, ejecutamos el siguiente comando:

\begin{lstlisting}
$ npm index.js 
\end{lstlisting}

Tras ejecutar este comando, si accedemos desde el navegador a la URL http://localhost:3000/ deberíamos ver el mensaje de "Hello world". Si podemos ver ese mensaje quiere decir que ya está montado el servidor.

    \item Abrir servidor a cualquier persona
    
Una vez tenemos el servidor montado, debemos abrirlo para el resto de usuarios, esto es posible con ngrok. Para utilizar ngrok se instala una extensión en Visual Studio Code (referencia a https://marketplace.visualstudio.com/items?itemName=philnash.ngrok-for-vscode).

Una vez instalada la extensión, dentro de VS Code tecleamos Ctrl+Shift+P para abrir la ventana de atajos de teclado, y tecleamos el siguiente comando:

\begin{lstlisting}
>ngrok: start 
\end{lstlisting}

Después de ejecutar este comando se nos pedirá indicar el puerto que usará el servidor. En mi caso el puerto a usar es el 3000.

Todo este proceso generará dinámicamente una URL para el servidor


\end{itemize}