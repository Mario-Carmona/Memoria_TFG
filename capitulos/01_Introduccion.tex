\chapter{Introducción}

La motivación de los seres humanos por difundir los conocimientos existentes a la mayor cantidad de personas posible ha existido desde que el ser humano es ser humano. Por supuesto, en sus inicios esta difusión se producía a una menor escala, principalmente a una escala local, de forma oral y a un ritmo muy lento. Además, eran conocimientos básicos, útiles sobre todo para la realización de las tareas diarias.

Conforme ha ido evolucionando el ser humano, esta difusión de conocimiento ha evolucionado junto a él. Por ejemplo, con hitos como la invención de la escritura, la difusión pudo aumentar su ritmo, tener un rango de alcance un poco mayor, y sobre todo posibilitó difundir conocimientos más complejos.

Aunque la evolución de la difusión de conocimiento se ha ido produciendo paulatinamente, el ritmo al que lo ha hecho no ha sido algo destacable, ya que siempre han existido dos grandes impedimentos. Por un lado, la falta de educación en la mayor parte de la población, lo que provoca la ausencia de motivación por aprender. Y, por otro lado, la dificultad de difusión, puesto que a pesar de haber tenido algunos avances gracias a hitos en la historia de la humanidad, tales como la invención de la escritura, mencionado anteriormente, o el aumento del tráfico mundial.

La falta de educación en la población se ha ido disipando en las últimas décadas, al hacer más fácil el acceso a los conocimientos.

El hito que si supuso un boom en la difusión de conocimiento fue la creación de la \gls{world_wide_web}. Este hito no solamente supuso un avance enorme en el aumento del tráfico de información mundial, sino también hizo aún más fácil el acceso de cualquier persona al conocimiento, sobre todo cuando se extendió el uso de ordenadores personales.

Después del último hito se ha llegado a tal punto de difusión, que el aumento de la difusión ha pasado a un segundo plano y ahora los esfuerzos se están centrando en hacer que la difusión sea más natural para aquellas personas que consultan la información.

Una de las posibles ramas para difundir conocimiento de forma natural es mediante la conversación con un \gls{bot}, o también llamados chatbots.

\section{Motivación y objetivos}

El potencial que tienen los chatbots como herramienta para la difusión de conocimientos de modo natural ha motivado la creación de este proyecto.

Además de lo anteriormente dicho, también ha motiva el desarrollo de este proyecto, la posible aplicación del mismo para los eventos de "La Noche Europea de l@s Investigador@s", en los cuales se realizan distintas actividades culturales. Una de estas actividades es el fomento sobre el mundo de la informática, acercando a todo tipo de personas a las posibilidades que tiene la informática de ser usada en las tareas que ellos llevan a cabo o que pretenden iniciar.

El principal objetivo de este proyecto es facilitar la creación de chatbots orientados a la difusión de información sobre una cierta temática y que además disponen de una cierta personalización a la hora de conversar dependiendo de ciertas características de la persona con la que se conversa.

\section{Descripción del proyecto}

El nombre que se le ha dado al chatbot es Sara. El nombre ha tenido como origen el nombre de la ciudad gls{Siracusa} en siciliano, Sarausa. El motivo de elegir este nombre es mi interés por la historia antigua. Además, idealmente el nombre de un chatbot debe ser corto y fácil de recordar, y el nombre de Sara lo cumple perfectamente.

El proyecto consiste en el desarrollo de una aplicación cliente-servidor. La aplicación es desplegada en la nube. Mi papel en este proyecto será parecida a la de un programador \gls{full_stack}.

En cuanto a la parte del cliente, es adaptable a cualquier tipo de \gls{front-end} o interfaz siguiendo un diseño \gls{responsive}. Esta parte en todo momento busca ser lo más cómoda y accesible posible para el usuario.

Pero la parte del servidor o \gls{back-end} es la más importante de esta aplicación, ya que contiene toda la lógica del chatbot. La parte del servidor es una \gls{API REST}, la cual permite la adaptabilidad de la interfaz, separando cliente y servidor.

El proyecto en su totalidad ha sido alojado en un repositorio de \gls{Github}, cuyo nombre es \href{https://github.com/Mario-Carmona/SARA_Chatbot}{SARA\_Chatbot}. Este repositorio se ha mantenido en privado durante el desarrollo del proyecto, pero tras su finalización se pondrá en público para su libre uso y consulta. Durante el desarrollo del proyecto se ha seguido una metodología de desarrollo ágil, la cual permite ir generando prototipos de la aplicación durante el desarrollo del proyecto, los cuales se pueden evaluar para tener en cuenta los fallos para futuros prototipos.

\section{Estructura de la memoria}

\begin{itemize}
\item \textbf{Capítulo 1. Introducción:} Breve introducción al proyecto, indicando además los motivos por los que se ha desarrollado y una breve descripción técnica del proyecto.
\end{itemize}

\section{Licencia}

********************************
