El servidor GPU está montado en Paperspace.

Se debe crear un proyecto, y dentro de este proyecto se crean notebooks.

Al crear los notebooks se debe elegir una plantilla, en mi caso al usar modelos de Huggingface, eligo la plantilla de Transformers + Huggingface.

Una vez se ha creado el notebook si se ejecuta el notebook dará un error de falta del archivo libcudart.so.11.0, dado que no está instalado CUDA.

Para instalar CUDA se han de seguir los siguientes pasos: → (https://medium.com/analytics-vidhya/install-cuda-11-2-cudnn-8-1-0-and-python-3-9-on-rtx3090-for-deep-learning-fcf96c95f7a1)

\begin{itemize}
    \item Comprobar la GPU que se está usando.
    
    Ejecutando el siguiente comando:
    
    \begin{lstlisting}
    nvidia-smi
    \end{lstlisting} 
    
    se observará la GPU asignada, la versión de los drivers instalada en la esquina superior izquierda, y la versión de CUDA que utiliza la GPU en la esquina superior derecha.
    
    Si al ejecutar el comando no aparece nada quiere decir que no están instalados los drivers de la GPU.
    
    \item (Opcional) Instalar los drivers
    
    Primero borramos cualquier drivers existente con el siguiente comando:
    
    \begin{lstlisting}
    sudo apt-get purge nvidia*
    \end{lstlisting} 
    
    Luego añadimos el repositorio donde obtener las librerías de los drivers, con el siguiente comando:
    
    \begin{lstlisting}
    sudo add-apt-repository ppa:graphics-drivers/ppa
    sudo apt-get update
    \end{lstlisting}
    
    Observamos los drivers disponibles para la GPU con el siguiente comando:
    
    \begin{lstlisting}
    ubuntu-drivers devices
    \end{lstlisting}
    
    Procedemos a instalar uno de los drivers.
    
    Se puede instalar automáticamente con el siguiente comando:
    
    \begin{lstlisting}
    ubuntu-drivers autoinstall
    \end{lstlisting}
    
    O se puede instalar un versión en concreto con el siguiente comando:
    
    \begin{lstlisting}
    sudo apt-get install <nombre de la versión>
    \end{lstlisting}
    
    \item Instalar CUDA 11.2
    
    \begin{itemize}
        \item Ejecutar script para instalar CUDA 11.2
        
        Se debe deseleccionar el instalar los drivers, y empezar a instalarlo
        
        \item Añadir variables de entorno
        
        Una vez instalado tal y como se indica en la salida del script anterior se deben añadir las siguientes variables de entorno al archivo ~/.bashrc .
        
        \begin{lstlisting}
        export PATH=/usr/local/cuda-11.2/bin${PATH:+:${PATH}}
        export LD_LIBRARY_PATH=/usr/local/cuda-11.2/lib64${LD_LIBRARY_PATH:+:${LD_LIBRARY_PATH}}
        export CUDA_HOME=/usr/local/cuda
        \end{lstlisting}
    \end{itemize}
    
    \item Instalar cuDNN 8.1.0
    
    El archivo comprimido se debe descargar de la página https://developer.nvidia.com/rdp/cudnn-archive.
    
    Una vez descargado y subido al servidor se debe descomprimir con el siguiente comando:
    
    \begin{lstlisting}
    tar -zvxf cudnn-11.2-linux-x64-v8.1.0.77.tgz
    \end{lstlisting}
    
    Al descomprimir el archivo obtenemos una carpeta con los include y la librerías de CUDA. Para que el sistema sepa donde se encuentra se añade la siguiente variable de entorno al archivo ~/.bashrc .
    
    \begin{lstlisting}
    export LD_LIBRARY_PATH=xxx/cuda/lib64:$LD_LIBRARY_PATH
    \end{lstlisting}
    
    Ejecutamos el archivo ~/.bashrc con el siguiente comando:
    
    \begin{lstlisting}
    source .bashrc
    \end{lstlisting}
    
    para activar todas las variables de entorno creadas.
    
    Damos permisos de lectura y escritura al archivo cudnn.h con el siguiente comando:
    
    \begin{lstlisting}
    sudo chmod a+r /usr/local/cuda/include/cudnn.h
    \end{lstlisting}
    
    \item Comprobar la instalación de CUDA
    
    Para comprobar que se ha instalado correctamente CUDA se utiliza el siguiente comando:
    
    \begin{lstlisting}
    nvcc -V
    \end{lstlisting}
    
    El cuál debe devolver una serie de información como la versión de CUDA instalada. Si no devuelve nada quiere decir que no se ha instalado correctamente.
\end{itemize}