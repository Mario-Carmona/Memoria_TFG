\chapter{Conclusiones y posibles ampliaciones}

El objetivo del proyecto era implementar un chatbot, con la capacidad de realizar acciones didácticas sobre un abanico de temas a través del establecimiento de una conversación con un usuario del sistema, y que pudiera ser usado en eventos como La Noche Europea de l@s Investigador@s.

Los conocimientos empleados para el desarrollo del proyecto están basados en los conocimientos adquiridos en un gran abanico de asignaturas del \href{https://grados.ugr.es/informatica/pages/infoacademica/estudios}{Plan de Estudios} del Grado en Ingeniería Informática como son \textit{Fundamentos de Bases de Datos, Fundamentos de Ingeniería del Software, Aprendizaje Automático, Procesadores de Lenguajes, Visión por computador, Programación Técnica y Científica y Estructuras de Datos}.

Basándonos en los resultados obtenidos en las pruebas del Apartado \ref{pruebas}, podríamos decir que se han efectuado todas las tareas de las historias de usuario planteadas en el Apartado \ref{sec:historias_usuario} para considerar que se han cumplido sus objetivos en su mayoría. Con la salvedad de la HU.7 (Tabla \ref{tab:HU7}), la cual hace referencia al comportamiento del chatbot a la hora de orientar la respuesta del mismo hacia la temática del chatbot de una forma didáctica. Basándonos en los resultados de las pruebas en la generación de respuestas, vemos como le cuesta al modelo realizar esta tarea didáctica, por lo tanto, no se puede afirmar el cumplimiento de esta historia de usuario. Aunque esto no quiere decir que el chatbot sea capaz de entablar una conversación con el usuario, la cual trascurre con cierta coherencia y no de forma aleatoria.

Pero quitando esta última historia de usuario, el resto de historias de usuario considero que se han cumplido con creces. Por ejemplo, la deducción de la edad del usuario se ha resuelto de manera sencilla y rápida mediante un modelo pre entrenado, que como se ha visto en las pruebas, da unos buenos resultados en la mayoría de situaciones. Por su puesto no siempre acierta, ya que en ocasiones el aspecto de una persona no está asociado con su edad real, esto forma parte de la complejidad de extraer información del aspecto humano. Aunque teniendo estos fallos, pienso que es una solución práctica, que evita la complejidad que podría suponer un análisis del lenguaje del usuario, para lo cual sería necesario un tiempo y unos conocimientos difícilmente alcanzables en el desarrollo de este proyecto. Y otro de los grandes logros es la obtención de una interfaz totalmente funcional y con ciertos elementos que la hagan un poco más atractiva como el modo oscuro y la interfaz responsive.

A pesar de que la HU.7 es una de las más importantes a la hora de desarrollar el proyecto, tal y como se indica en la Tabla \ref{tab:listado_HU}; no se puede considerar que el proyecto haya sido todo un fracaso, ya que el desarrollo de un chatbot tan complejo es una tarea muy difícil. Por supuesto, este fracaso en el cumplimiento de la HU.7 viene derivado de la materialización del riesgo de fijar objetivos pocos realistas. Está claro que no es un objetivo imposible, pero si difícilmente alcanzable en el periodo de tiempo disponible para el desarrollo de este proyecto.

Algo que hay que recalcar es que aunque no se haya conseguido un chatbot didáctico, no quiere decir que no se haya trabajado en el proyecto. Como ya se indicó en el apartado de gestión, se han invertido un número mayor de horas de las que requiere como mínimo un TFG. Durante estas horas se han adquirido una cantidad ingente de conocimientos, ya que casi todas las tecnologías usadas en el proyecto eran desconocidas, exceptuando la plataforma Github, ciertos conocimientos sobre los contenedores, y la herramienta Overleaf, así como una base en el lenguaje \textit{LaTeX}. Pero, por otro lado, he realizado tareas cuyos conocimientos requeridos no habían sido vistos en todo el grado.

Una de las primeras tareas que me hizo ponerme a investigar sobre el ámbito de las páginas web, fue la elaboración de una interfaz web para nuestro sistema. Durante esta investigación se adquirieron muchos conocimientos en el lenguaje HTML, Javascript y CSS. Y la adquisición de conocimientos no se quedó en lo teórico, sino que se puso en práctica enlazando elementos implementados en los distintos lenguajes y obteniendo un producto final con grandes posibilidades.

Otro punto que me empujó a investigar mucho fue la implementación del servidor del Controlador, ya que tuve que utilizar el framework FastAPI, el cual era totalmente desconocido para mí. Estos conocimientos también me ayudaron con el servidor del Modelo porque se basan en el mismo framework.

Y por último, una de las partes que aunque en principio parezca que no llevan mucho trabajo, internamente si lo llevan. Esta parte son los modelos empleados para realizar la deducción de edad o la generación de respuestas. Aunque se hayan usado modelos pre entrenados, se han utilizado modelos basados en Transformers, los cuales no había utilizado nunca porque solamente había usado modelos basados en convoluciones para la asignatura de \textit{Visión por computador}. Además, son modelos implementados con PyTorch y hasta el momento únicamente había trabajado con TensorFlow. Y también que sean modelo pre entrenados no implica que no haya que trabajar con ellos y, por tanto, se debe aprender como funcionan estos modelos y por supuesto, como funciona la plataforma Huggingface que es la que contiene los modelos pre entrenados.

En definitiva, este proyecto ha sido todo un reto para mí, ya que se han puesto a prueba todos mis conocimientos adquiridos durante el grado y capacidades para adaptarme, puesto que he desarrollado un sistema desde la parte más visible como es el frontend y la parte más oculta y lógica del sistema como es el backend.

Por último, la reflexión final es que el sistema implementado es un sistema que cumple con muchos de sus objetivos, pero que sigue teniendo mucho potencial por explotar si se invierte un poco más de tiempo en él.

\section{Posibles ampliaciones}

Como se ha ido indicando en esta parte final de la memoria, el proyecto tiene claras vías por donde ampliar y mejorar el proyecto. Algunas de las posibles ampliaciones son las siguientes:

\begin{itemize}
\item \textbf{Búsqueda y utilización de una plataforma de despliegue con GPU:} Si utilizamos una plataforma que tenga las características y las funcionalidades que tiene Heroku, junto con la posibilidad de utilizar una GPU, podríamos realizar una reestructuración del sistema la cual simplificaría esta estructura. La simplificación conllevaría la unión de los servidores del Controlador y del Modelo, y una refactorización completa del código. Esta unión evita el trabajo extra derivado de implementar dos servidores y de tener que mantenerlos en funcionamiento y actualizados por separado, teniendo partes en común. Además, la unión elimina muchos envíos de peticiones entre servidores, ya que el intercambio de información entre el Controlador y el Modelo se efectuarían de forma interna, lo que conlleva mayor rapidez y seguridad.
\item \textbf{Mejora de la interfaz web:} Basándonos en el feedback de los usuarios que ejecutaron las pruebas de la interfaz web, se podría modernizar la interfaz web empleando herramientas más actuales para implementar la funcionalidad de la interfaz, y que esta tome un aspecto más renovado.
\item \textbf{Seguir con la obtención de información:} Como se ha indicado en varios de los apartados de la memoria, es necesario avanzar en la obtención de información para intentar que el chatbot adquiera un tono didáctico.
\item \textbf{Reutilización de los resultados:} Un chatbot nunca deja de mejorarse, ya que normalmente se reutilizan los resultados de conversaciones realizadas con el chatbot para mejorar al mismo. Nuestro sistema ya es capaz de guardar las conversaciones realizadas con el chatbot en una base de datos, pero nos falta un programa que sea capaz de mover esa información y entrenar al chatbot con ella. Además, sería muy interesante que se puede automatizar este proceso, ya que evitaría mucha carga de trabajo para el administrador del sistema.
\item \textbf{Mejorar el modelo de deducción de edad:} Aunque como se ha visto en las pruebas del modelo de deducción tiene una calidad más que aceptable para nuestro sistema, siempre se puede mejorar porque se trata de un modelo pre entrenado. Por lo tanto, podríamos entrenar al modelo con muchas más imágenes que podemos obtener de las múltiples bases de imágenes existentes en internet. Claramente, esto conlleva una carga de trabajo tremenda, ya que se deben obtener las imágenes, etiquetarlas si no lo están, y por último crear un programa para el entrenamiento del modelo pre entrenado.
\item \textbf{Creación de una interfaz propia para el chatbot:} La creación de una interfaz propia podría ayudar a seguir avanzando en la simplificación del sistema, por el hecho de que ahora mismo se deben generar secciones distintas para cada rango de edad que distingue el sistema. Estas duplicidades vienen derivadas del uso de la interfaz de Dialogflow. Por lo tanto, una gran mejora sería crear nuestra propia interfaz web para eliminar todas estas duplicidades y poder simplificar el sistema. Además, tendríamos un mayor control del sistema. Incluso se podría dejar de hacer uso de Dialogflow, aunque esto conllevaría un mayor trabajo dado que eliminamos las facilidades que nos da Dialogflow a la hora de desplegar el chatbot.
\item \textbf{Uso de un avatar:} Actualmente se está poniendo de moda producir chatbots con una interfaz visual compuesta por un avatar. Este avatar tiene un aspecto humano para hacer más cómodo el establecimiento de las conversaciones. Con este avatar eliminamos la idea antigua de tener una interfaz textual para poder interactuar con el chatbot.
\end{itemize}

