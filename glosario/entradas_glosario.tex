
\makeglossaries


\newglossaryentry{world_wide_web}
{
    name=World Wide Web,
    description={La World Wide Web (W3) fue desarrollada por un conjunto de personas, de entre las que destaca Tim Berners-Lee, a finales de la década de los ochenta. El motivo por el que surgió este desarrollo fue el interés por parte del CERN de poder compartir sus investigaciones con el resto del mundo y por tanto mejorar la comunicación en el mundo investigador.}
}

\newglossaryentry{bot}
{
    name=bot,
    description={Un bot \footnote{https://dictionary.cambridge.org/es-LA/dictionary/english/bot} es un programa informático que trabaja de forma autónoma.}
}

\newglossaryentry{chatbot}
{
    name=chatbot,
    description={Un chatbot \footnote{https://dictionary.cambridge.org/es-LA/dictionary/english/chatbot} es un programa informático diseñado para tener una conversación con un humano, especialmente a través de Internet.}
}

\newglossaryentry{front-end}
{
    name=front-end,
    description={El front-end es la capa que se encuentra por encima del \gls{back-end}. Es la capa visible para el usuario y también es con la que interactúa. Debido a se interactúa con ella debe cumplir unos estándares de usabilidad y estética. Los principales lenguajes de programación con los que se implementa esta capa son: Javascript, CSS y HTML.}
}

\newglossaryentry{back-end}
{
    name=back-end,
    description={El back-end es la capa que se encuentra por debajo del \gls{front-end}. Es la capa no visible para el usuario. En ella se encuentra el funcionamiento de la aplicación, incluyendo servidores y bases de datos. No existe un lenguaje de programación específico para esta capa.}
}

\newglossaryentry{full_stack}
{
    name=Full Stack,
    description={La programación Full Stack abarca tanto la programación del \gls{front-end} como la del \gls{back-end}. Los profesionales de esta programación se denominan programadores Full Stack y tienen las habilidades necesarias para administrar un proyecto completo.}
}

\newglossaryentry{responsive}
{
    name=responsive,
    description={Según Brett S. Gardner \cite{RefWorks:RefID:34-2011responsive}, el diseño responsive de las páginas web permite crear una única página web que es capaz de adaptarse a la interfaz y al contenido de la misma para poder ser visible en distintos dispositivos (móviles, tablets y ordenadores). Este diseño permite mejorar la experiencia del usuario. Actualmente el diseño responsive se implementa a través del lenguaje CSS, en concreto con su versión CSS3, la cuál tiene un enfoque centrado en el usuario permitiendo el diseño de aplicaciones web dinámicas.}
}

\newglossaryentry{API REST}
{
    name=API REST,
    description={Una API REST es un \gls{API} que sigue los principios de diseño de REST. El principal objetivo de estos principios es tener una \gls{API} que posibilite la separación entre cliente y servidor, para ello algunos de sus principios son por ejemplo: tener un interfaz uniforme, es decir, todas las solicitudes hacia un recurso de la API son siempre iguales; o no tener estado, es decir, la API no guarda información, todas la información necesaria para resolver la solicitud debe estar contenida en la misma.}
}

\newglossaryentry{API}
{
    name=API,
    description={Según la página web de IBM \cite{RefWorks:RefID:33-2021-API}, una API o interfaz de programación de aplicaciones, es un conjunto de reglas que determinan cómo las aplicaciones o los dispositivos pueden conectarse y comunicarse entre sí.}
}

\newglossaryentry{Siracusa}
{
    name=Siracusa,
    description={Siracusa es una ciudad de Italia, situada en la costa sudeste de la isla de Sicilia, famosa como centro cultural desde la Antigua Grecia.}
}

\newglossaryentry{Github}
{
    name=Github,
    description={Github es una plataforma donde multitud de personas y empresas desarrollan software. Esta plataforma permite la comunicación e interacción de la comunidad de desarrolladores de software. Los proyectos se alojan en lo que se denominan repositorios, los cuáles pueden ser públicos o privados. Cada repositorio tiene un control de versiones mediante Git.}
}


