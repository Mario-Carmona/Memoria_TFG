
\makeglossaries


\newglossaryentry{world_wide_web}
{
    name=World Wide Web,
    description={La World Wide Web (W3) fue desarrollada por un conjunto de personas, de entre las que destaca Tim Berners-Lee, a finales de la década de los ochenta. El motivo por el que surgió este desarrollo fue el interés por parte del CERN de poder compartir sus investigaciones con el resto del mundo y por tanto mejorar la comunicación en el mundo investigador}
}

\newglossaryentry{bot}
{
    name=bot,
    description={Un bot \footnote{https://dictionary.cambridge.org/es-LA/dictionary/english/bot} es un programa informático que trabaja de forma autónoma}
}

\newglossaryentry{chatbot}
{
    name=chatbot,
    description={Un chatbot \footnote{https://dictionary.cambridge.org/es-LA/dictionary/english/chatbot} es un programa informático diseñado para tener una conversación con un humano, especialmente a través de Internet}
}

\newglossaryentry{front-end}
{
    name=front-end,
    description={El front-end es la capa que se encuentra por encima del \gls{back-end}. Es la capa visible para el usuario y también es con la que interactúa. Debido a se interactúa con ella debe cumplir unos estándares de usabilidad y estética. Los principales lenguajes de programación con los que se implementa esta capa son: Javascript, CSS y HTML}
}

\newglossaryentry{back-end}
{
    name=back-end,
    description={El back-end es la capa que se encuentra por debajo del \gls{front-end}. Es la capa no visible para el usuario. En ella se encuentra el funcionamiento de la aplicación, incluyendo servidores y bases de datos. No existe un lenguaje de programación específico para esta capa}
}

\newglossaryentry{full_stack}
{
    name=Full Stack,
    description={La programación Full Stack abarca tanto la programación del \gls{front-end} como la del \gls{back-end}. Los profesionales de esta programación se denominan programadores Full Stack y tienen las habilidades necesarias para administrar un proyecto completo}
}

\newglossaryentry{responsive}
{
    name=responsive,
    description={Según Brett S. Gardner \cite{RefWorks:RefID:34-2011responsive}, el diseño responsive de las páginas web permite crear una única página web que es capaz de adaptarse a la interfaz y al contenido de la misma para poder ser visible en distintos dispositivos (móviles, tablets y ordenadores). Este diseño permite mejorar la experiencia del usuario. Actualmente el diseño responsive se implementa a través del lenguaje CSS, en concreto con su versión CSS3, la cuál tiene un enfoque centrado en el usuario permitiendo el diseño de aplicaciones web dinámicas}
}

\newglossaryentry{API REST}
{
    name=API REST,
    description={Una API REST es un \gls{API} que sigue los principios de diseño de REST. El principal objetivo de estos principios es tener una \gls{API} que posibilite la separación entre cliente y servidor, para ello algunos de sus principios son por ejemplo: tener un interfaz uniforme, es decir, todas las solicitudes hacia un recurso de la API son siempre iguales; o no tener estado, es decir, la API no guarda información, todas la información necesaria para resolver la solicitud debe estar contenida en la misma}
}

\newglossaryentry{API}
{
    name=API,
    description={Según la página web de IBM \cite{RefWorks:RefID:33-2021-API}, una API o interfaz de programación de aplicaciones, es un conjunto de reglas que determinan cómo las aplicaciones o los dispositivos pueden conectarse y comunicarse entre sí}
}

\newglossaryentry{Siracusa}
{
    name=Siracusa,
    description={Siracusa es una ciudad de Italia, situada en la costa sudeste de la isla de Sicilia, famosa como centro cultural desde la Antigua Grecia}
}

\newglossaryentry{Github}
{
    name=Github,
    description={Github es una plataforma donde multitud de personas y empresas desarrollan software. Esta plataforma permite la comunicación e interacción de la comunidad de desarrolladores de software. Los proyectos se alojan en lo que se denominan repositorios, los cuáles pueden ser públicos o privados. Cada repositorio tiene un control de versiones mediante Git}
}


\newglossaryentry{IA}
{
    name=inteligencia artificial,
    description={La inteligencia artificial (IA) es, en ciencias de la computación, la disciplina que intenta replicar y desarrollar la inteligencia y sus procesos implícitos a través de computadoras}
}

\newglossaryentry{NLU}
{
    name=técnicas avanzadas de comprensión del lenguaje natural,
    description={La comprensión de lenguaje natural es una parte del procesamiento de lengua natural en inteligencia artificial que trabaja con la comprensión de lectura en dispositivos electrónicos}
}

\newglossaryentry{DST}
{
    name=DST,
    description={El módulo DST (Dialogue State Tracker) es el encargado de ir estimando el objetivo del usuario a medida que avanza el diálogo}
}

\newglossaryentry{DPO}
{
    name=DPO,
    description={El módulo DPO (Dialogue Policy Optimization) es el encargado de aplicar las políticas de optimización sobre el chatbot}
}

\newglossaryentry{Markov}
{
    name=proceso de decisión de Markov,
    description={Son las forma idealizada matemáticamente del problema de aprendizaje por refuerzo. Estos procesos describen formalmente el medio ambiente en el cual se desarrolla el aprendizaje por refuerzo}
}

\newglossaryentry{GAN}
{
    name=redes generativas adversativas,
    description={Son una clase de algoritmos de inteligencia artificial que se utilizan en el aprendizaje no supervisado, implementados por un sistema de dos redes neuronales que compiten mutuamente en una especie de juego de suma cero}
}

\newglossaryentry{IoT}
{
    name=internet de las cosas,
    description={El Internet de las cosas (IoT) es el proceso que permite conectar elementos físicos cotidianos al Internet}
}

\newglossaryentry{prototipo}
{
    name=prototipo,
    description={Un prototipo es un primer modelo que sirve como representación o simulación del producto final y que nos permite verificar el diseño y confirmar que cuenta con las características específicas planteadas}
}

\newglossaryentry{TIC}
{
    name=TIC,
    description={Conjunto de técnicas y equipos informáticos que permiten comunicarse a distancia por vía electrónica}
}

\newglossaryentry{backup}
{
    name=backup,
    description={Es una copia de seguridad de los datos. Al hacer un backup, se pueden restaurar los datos posteriormente en caso de pérdida}
}

\newglossaryentry{feedback}
{
    name=feedback,
    description={El feedback (retroalimentación) es la acción de ofrecer información a una persona sobre un resultado}
}

\newglossaryentry{ECTS}
{
    name=ECTS,
    description={ECTS es la sigla correspondiente al European Credit Transder System (Sistema Europeo de Transferencia de Créditos) y es el sistema adoptado por todas las universidades del Espacio Europeo de Educación Superior (EEES) para garantizar la homogeneidad y la calidad de los estudios que ofrecen}
}

\newglossaryentry{link}
{
    name=link,
    description={Un link es un elemento que establece un vínculo con otro recurso}
}

\newglossaryentry{PyTorch}
{
    name=PyTorch,
    description={PyTorch es una biblioteca de aprendizaje automático​ de código abierto basada en la biblioteca de Torch}
}

\newglossaryentry{Transformers}
{
    name=Transformers,
    description={Se trata de una arquitectura de redes neuronales que en estos momentos se considera estado del arte en modelos secuenciales}
}

\newglossaryentry{plugin}
{
    name=plugin,
    description={Los plugins son complementos que añaden funcionalidades extra o mejoras a los programas}
}

\newglossaryentry{tip}
{
    name=tip,
    description={Tip es un término inglés que puede traducirse como “consejo” o “sugerencia”}
}

\newglossaryentry{PaaS}
{
    name=PaaS,
    description={PaaS significa Plataforma como servicio. Se trata de un conjunto de servicios basados en la nube que permiten a los usuarios empresariales y desarrolladores crear aplicaciones de forma rápida y rentable}
}

\newglossaryentry{canvas}
{
    name=canvas,
    description={Canvas (lienzo) es un elemento HTML que permite la generación de gráficos y animaciones de forma dinámica por medio de scripts tras elegir el dominio para tu web}
}

\newglossaryentry{Add-ons}
{
    name=Add-ons,
    description={Un add-on es un programa o módulo adicional que complementa navegadores, programas de correo y otras aplicaciones de software y hardware y añade funciones que el sistema básico no tiene}
}

\newglossaryentry{framework}
{
    name=framework,
    description={Un framework es un esquema o marco de trabajo que ofrece una estructura base para elaborar un proyecto con objetivos específicos, una especie de plantilla que sirve como punto de partida para la organización y desarrollo de software}
}

\newglossaryentry{cloud}
{
    name=cloud,
    description={Cloud significa, literalmente, nube. En términos informáticos nos referimos a un paradigma que permite ofrecer servicios de computación a través de una red, que normalmente es Internet}
}

\newglossaryentry{portfolio}
{
    name=portfolio,
    description={Un portfolio es un conjunto de elementos}
}

\newglossaryentry{webhook}
{
    name=webhook,
    description={Los webhooks son herramientas para que las diferentes aplicaciones puedan comunicarse entre sí}
}

\newglossaryentry{CNN}
{
    name=CNN,
    description={Una red neuronal convolucional (CNN) es una arquitectura de red para deep learning que aprende directamente de los datos, sin necesidad de extraer características manualmente}
}

\newglossaryentry{mockup}
{
    name=mockup,
    description={Un mockup, traducido del inglés como bosquejo, es un fotomontaje a través del cual los diseñadores gráficos pueden presentar sus propuestas a los clientes}
}

\newglossaryentry{epocas}
{
    name=épocas,
    description={Recorrido completo de todos los datos de entrenamiento}
}
