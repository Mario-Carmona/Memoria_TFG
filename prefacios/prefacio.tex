\chapter*{}
%\thispagestyle{empty}
%\cleardoublepage

%\thispagestyle{empty}

%\cleardoublepage
\thispagestyle{empty}

\begin{center}
{\large\bfseries Agente conversacional multiedad: Agente conversacional adaptable}\\
\end{center}
\begin{center}
Mario Carmona Segovia\\
\end{center}

%\vspace{0.7cm}
\noindent{\textbf{Palabras clave}: Agentes conversacionales, Aprendizaje automático, Transformers, BlenderBot, SCRUM, Arquitectura MVC, Heroku, Responsive, FastAPI.}\\

\vspace{0.7cm}
\noindent{\textbf{Resumen}}\\

Este proyecto tiene como objetivo crear de forma automatizada un agente conversacional para ``La noche europea de los investigadores'' (ERN). Deberá poder contestar a las preguntas que le realicen los asistentes, adaptándose a sus edades y conocimientos de forma interactiva.

Para el desarrollo de este proyecto se ha construido un sistema con el que poder hacer uso del chatbot obtenido del entrenamiento. Además, se ha elaborado toda una estructura de programas para obtener información, entrenar a los modelos con esa información y finalmente desplegar estos modelos en el sistema mencionado anteriormente.

Para la adaptación del chatbot a la edad se ha elaborado un módulo que es capaz de deducir la edad de un usuario de forma explícita o de manera implícita a través de imágenes.
\cleardoublepage


\thispagestyle{empty}


\begin{center}
{\large\bfseries Multi-age conversational agent: Teaching-focused conversational agent}\\
\end{center}
\begin{center}
Mario Carmona Segovia\\
\end{center}

%\vspace{0.7cm}
\noindent{\textbf{Keywords}: Conversational Agents, Machine Learning, Transformers, BlenderBot, SCRUM, MVC Architecture, Heroku, Responsive, FastAPI.}\\

\vspace{0.7cm}
\noindent{\textbf{Abstract}}\\

This project aims to create an automated conversational agent for ``The European Researchers Night'' (ERN). It should be able to answer the questions asked by the attendees, adapting to their ages and knowledge in an interactive way.

For the development of this project, a system has been built to make use of the chatbot obtained from the training. In addition, a whole structure of programs has been developed to obtain information, train the models with this information and finally deploy these models in the aforementioned system.

In order to adapt the chatbot to age, a module has been developed that is able to deduce the age of a user explicitly or implicitly through images.

\chapter*{}
\thispagestyle{empty}

\noindent\rule[-1ex]{\textwidth}{2pt}\\[4.5ex]

Yo, \textbf{Mario Carmona Segovia}, alumno de la titulación Grado en Ingeniería Informática de la \textbf{Escuela Técnica Superior
de Ingenierías Informática y de Telecomunicación de la Universidad de Granada}, con DNI 45922466E, autorizo la
ubicación de la siguiente copia de mi Trabajo Fin de Grado en la biblioteca del centro para que pueda ser
consultada por las personas que lo deseen.

\vspace{6cm}

\noindent Fdo: Mario Carmona Segovia

\vspace{2cm}

\begin{flushright}
Granada a 7 de julio de 2022 .
\end{flushright}


\chapter*{}
\thispagestyle{empty}

\noindent\rule[-1ex]{\textwidth}{2pt}\\[4.5ex]

D. \textbf{Rocío Celeste Romero Zaliz}, Profesora del Departamento de Ciencias de la Computación e Inteligencia Artificial de la Universidad de Granada.

\vspace{0.5cm}

\textbf{Informan:}

\vspace{0.5cm}

Que el presente trabajo, titulado \textit{\textbf{Agente conversacional multiedad, Agente conversacional adaptable}},
ha sido realizado bajo su supervisión por \textbf{Mario Carmona Segovia}, y autorizamos la defensa de dicho trabajo ante el tribunal
que corresponda.

\vspace{0.5cm}

Y para que conste, expiden y firman el presente informe en Granada a 7 de julio de 2022 .

\vspace{1cm}

\textbf{La directora:}

\vspace{5cm}

\noindent \textbf{Rocío Celeste Romero Zaliz}

\chapter*{Agradecimientos}
\thispagestyle{empty}

       \vspace{1cm}


En primer lugar, me gustaría dar las gracias a la Universidad de Granada y a toda su comunidad estudiantil que ha hecho del trascurso de la carrera toda una experiencia que me ha convertido en la persona que soy a día de hoy. Por supuesto, me gustaría hacer especial énfasis en todos aquellos amigos de la carrera, Dani, Sergio, Óscar, Paco, Víctor, Pedro y un sin fin más; con lo que he compartido los momentos más felices y también los más tristes, y sin olvidar todos los momentos graciosos que hemos vivido, ya que cuatro años dan para muchas experiencias. No me gustaría dejar por alto a mi querida tutora Rocío, la cual me ha dado mucha confianza a lo largo de todo el proyecto y me ha apoyado en todos aquellos momentos donde el proyecto se ha puesto cuesta arriba.

Y finalmente, está claro que un apoyo fundamental ha sido toda mi familia, la cual no solamente me ha apoyado emocionalmente, sino que si ha hecho falta me ha ayudado con ideas para avanzar en el proyecto.
